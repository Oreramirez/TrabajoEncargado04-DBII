%
\documentclass[%
 reprint,
 amsmath,amssymb,
 aps,
]{revtex4-1}

\usepackage{graphicx}% Include figure files
\usepackage{dcolumn}% Align table columns on decimal point
\usepackage{bm}% bold math


\begin{document}


\title{Universidad Privada de Tacna}
\title{Escuela Profesional de Ingenieria de Sistemas}
\title{VIRTUALIZACIÓN Y CONTENEDORES}
\author{Yofer Nain Catari Cabrera}

\affiliation{%
 Universidad Privada de Tacna \textbackslash Facultad de Ingenieria \textbackslash Escuela Profesional de Ingenieria de Sistemas
}%

\begin{abstract}
\begin{center}
\textbf{Resumen}
\end{center}
En este trabajo se realizará un estudio de las tecnologías de virtualización de
contenedores con el fin de implementar y poner en marcha un sistema que permita
orquestar el despliegue de aplicaciones sobre un entorno empresarial.
Para ello, se realizará en primera instancia un análisis de los sistemas de virtualización
más habituales para continuar introduciendo los conceptos y sistemas de virtualización
de contenedores. Una vez introducida la parte teórica se analizan distintas
herramientas de virtualización de contenedores centrándonos en la herramienta
Docker para la cual se detalla su arquitectura, funcionamiento y proceso de instalación
para finalizar con un par de ejemplos prácticos de despliegue de servicios.
A continuación, una vez que ya hemos implementado y analizado un sistema de
virtualización de contenedores como tecnología necesaria de base, pasamos a
estudiar distintas soluciones del mercado para implementar un sistemas de
orquestación basado en microservicios para el despliegue de aplicaciones de carácter
corporativo. Finalizamos con la implantación, instalación y puesta en marcha del
sistema estudiado acompañado de unos ejemplos de orquestación usando dos
aplicaciones de código abierto que se ven bastante habitualmente en los entornos
corporativos actuales para dar soporte a distintas soluciones.
.\\
\textbf{Palabras clave:}   Virtualizacion y Contenedores .\\

\begin{center}
\textbf{Abstract}
\end{center}
In this job we do a study of container virtualization technologies will be carried out in
order to implement a system that allows orchestrating the deployment of applications
and services in enterprise environments.
This will be done in the first instance an analysis of the most common virtualization
systems to continue introducing concepts and systems of containerization
virtualization. Once the theoretical part is introduced, different container virtualization
tools are analyzed, focusing on the Docker tool, which details its architecture, operation
and installation process to finish with a couple of practical examples of service
deployment.
Then, once we have already implemented and analyzed a container virtualization
system and all necessary background technology, we started to study different
solutions in the market to implement a micro-services based orchestration systems for
the deployment of corporate applications.
Then finish with timplementation, installation and first steps of the studied system
accompanied by some examples of orchestration using two open source applications
that are quite commonly used in the current corporate environments to support
different solutions.
\\
\textbf{Keywords:}   Virtualization and Containers

\end{abstract}



\maketitle

%\tableofcontents

\section {Introducción}\label{sec:1}
La tecnología que ha llegado para complementar y completar la virtualización de servidores es la
virtualización de contenedores de aplicaciones. Esta tecnología va un paso mas allá en el
paradigma de la virtualización, permitiendo no sólo el salto de virtualizar servidores sino también
de virtualizar directamente un contenedor donde se ejecuta una aplicación, permitiendo de este
modo una mayor abstracción aislando la componente "lógica de la aplicación" del componente
“sistema operativo”.


%-----------------------------------------------------------------
\section{Objetivos}\label{sec:2}

\subsection{General:}

-  
\subsection{Específicos:}
- 

%-----------------------------------------------------------------
\section {Marco Teórico}\label{sec:3}


\subsection{Máquinas Virtuales}
\subsection{Contenedores}
\subsection{Diferencias}

%-----------------------------------------------------------------

%-----------------------------------------------------------------
\section{tipos de Virtualizacion}\label{sec:4}
\subsection{¿Cuales son las Ventajas de la Virtualizacion}
-  
\subsection{Contenedores}
- 

%-----------------------------------------------------------------
\section {Virtualizacion de Contenedores}\label{sec:5}


\subsection{Diferencia Entre Virtualizacion Clasica y Virualizacion de Contenedores}
\subsection{Tecnologia a Utilizar}
\subsection{Presedentes a Utilizar}
\subsection{Ventajas y Desventajas}

%-----------------------------------------------------------------

%-----------------------------------------------------------------
\section{Contenedores Docker}\label{sec:6}
\subsection{Orquesta de Aplicacion}
-  
\subsection{Docker y Otros Container : Mas alla de la virtualizacion}
- 
\subsection{Contenedor Docker la Tecnologia de Contenedores a mano}
\subsection{Ventajas de la Tecnologia de Contenedores}
\subsection{¿Son Seguros los Contenedores?}

%-----------------------------------------------------------------
\section {Hypervisores Base Metal}\label{sec:7}


\subsection{Tipos de Hypervisores}
\subsection{Ventajas y Desventajas}
\subsection{Productos Comerciales}

%-----------------------------------------------------------------
\section{Conclusiones}\label{sec:8}


\begin{itemize}
	\item El mapeo objeto-relacional (ORM) es una técnica para mapear sistemas orientados a objetos a bases de datos relacionales.
	\item Utilizar las diferentes herramientas de ORM, da mas rapidez y eficiencia a la construcción de base de datos.
\end{itemize}

\section{webgrafia}\label{sec:9}
% Bibliografia.
%-----------------------------------------------------------------

\bibliographystyle{plain}
\bibliography{Bibliografia}

\end{document}