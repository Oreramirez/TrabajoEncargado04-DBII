%
\documentclass[%
 reprint,
 amsmath,amssymb,
 aps,
]{revtex4-1}

\usepackage{graphicx}% Include figure files
\usepackage{dcolumn}% Align table columns on decimal point
\usepackage{bm}% bold math


\begin{document}


\title{Universidad Privada de Tacna}
\title{Escuela Profesional de Ingenieria de Sistemas}
\title{VIRTUALIZACIÓN Y CONTENEDORES}
\author{Yofer Nain Catari Cabrera}

\affiliation{%
 Universidad Privada de Tacna \textbackslash Facultad de Ingenieria \textbackslash Escuela Profesional de Ingenieria de Sistemas
}%

\begin{abstract}
\begin{center}
\textbf{Resumen}
\end{center}
En este artículo aprenderemos sobre el concepto y sus herramientas de mapeo objeto-relacional que nos permitirá crear una capa de acceso a los datos de una base de datos relacional, de tal forma que las tablas se transforman en clases y las filas de las tablas en objetos.\\
\textbf{Palabras clave:}   ORM, base de datos, herramientas, clases, objetos.\\

\begin{center}
\textbf{Abstract}
\end{center}
In this article we learn about concept and tools of Object-Relational mapping ORM that we will allow us to create an access layer to the data of a relational database, in such a way that the tables are transformed into classes and the rows of the table into objects.\\
\textbf{Keywords:}   ORM, database, tools, classes, objects.

\end{abstract}



\maketitle

%\tableofcontents

\section {Introducción}\label{sec:1}
Existen una gran variedad de herramientas para realizar el Mapeo Objeto-Relacional, ya sean libres como de pago. Algunas de estas estan ligadas al lenguaje de programacion que es orientada a objetos. Para minimizar estas dificultades de persistencia de datos, existen herramientas que se encargan de generar de manera automática el acceso a los datos, abstrayendo al programador de esta tarea. Son las llamadas herramientas ORM.
Estas herramientas que se mencionarán en el articulo, tanto para Java, .Net, PHP y Python, son una buena solucion para el desarrollo de aplicaciones de programacion orientadas a objetos (POO). 
El aprendizaje del lenguaje de la herramienta ORM puede ser algo complejo ya que, para poder sacar el máximo provecho a la herramienta, es necesario conocer en profundidad cómo funciona la misma.


%-----------------------------------------------------------------
\section{Objetivos}\label{sec:2}
\subsection{General:}
-  
\subsection{Específicos:}
- 

%-----------------------------------------------------------------
\section {Marco Teórico}\label{sec:3}


\subsection{Máquinas Virtuales}
\subsection{Contenedores}
\subsection{Diferencias}

%-----------------------------------------------------------------

%-----------------------------------------------------------------
\section{tipos de Virtualizacion}\label{sec:4}
\subsection{¿Cuales son las Ventajas de la Virtualizacion}
-  
\subsection{Contenedores}
- 

%-----------------------------------------------------------------
\section {Virtualizacion de Contenedores}\label{sec:5}


\subsection{Diferencia Entre Virtualizacion Clasica y Virualizacion de Contenedores}
\subsection{Tecnologia a Utilizar}
\subsection{Presedentes a Utilizar}
\subsection{Ventajas y Desventajas}

%-----------------------------------------------------------------

%-----------------------------------------------------------------
\section{Contenedores Docker}\label{sec:6}
\subsection{Orquesta de Aplicacion}
-  
\subsection{Docker y Otros Container : Mas alla de la virtualizacion}
- 
\subsection{Contenedor Docker la Tecnologia de Contenedores a mano}
\subsection{Ventajas de la Tecnologia de Contenedores}
\subsection{¿Son Seguros los Contenedores?}

%-----------------------------------------------------------------
\section {Hypervisores Base Metal}\label{sec:7}


\subsection{Tipos de Hypervisores}
\subsection{Ventajas y Desventajas}
\subsection{Productos Comerciales}

%-----------------------------------------------------------------
\section{Conclusiones}\label{sec:8}


\begin{itemize}
	\item El mapeo objeto-relacional (ORM) es una técnica para mapear sistemas orientados a objetos a bases de datos relacionales.
	\item Utilizar las diferentes herramientas de ORM, da mas rapidez y eficiencia a la construcción de base de datos.
\end{itemize}

\section{webgrafia}\label{sec:9}
% Bibliografia.
%-----------------------------------------------------------------

\bibliographystyle{plain}
\bibliography{Bibliografia}

\end{document}