%
\documentclass[%
 reprint,
 amsmath,amssymb,
 aps,
]{revtex4-1}

\usepackage{graphicx}% Include figure files
\usepackage{dcolumn}% Align table columns on decimal point
\usepackage{bm}% bold math


\begin{document}


\title{Universidad Privada de Tacna}
\title{Escuela Profesional de Ingenieria de Sistemas}
\title{VIRTUALIZACIÓN Y CONTENEDORES}
\author{Yofer Nain Catari Cabrera}
\author{Orestes Ramirez Ticona}
\affiliation{%
 Universidad Privada de Tacna \textbackslash Facultad de Ingenieria \textbackslash Escuela Profesional de Ingenieria de Sistemas
}%

\begin{abstract}
\begin{center}
\textbf{Resumen}
\end{center}
En este trabajo se realizará un estudio de las tecnologías de virtualización de
contenedores con el fin de implementar y poner en marcha un sistema que permita
orquestar el despliegue de aplicaciones sobre un entorno empresarial.
Para ello, se realizará en primera instancia un análisis de los sistemas de virtualización
más habituales para continuar introduciendo los conceptos y sistemas de virtualización
de contenedores. Una vez introducida la parte teórica se analizan distintas
herramientas de virtualización de contenedores centrándonos en la herramienta
Docker para la cual se detalla su arquitectura, funcionamiento y proceso de instalación
para finalizar con un par de ejemplos prácticos de despliegue de servicios.
A continuación, una vez que ya hemos implementado y analizado un sistema de
virtualización de contenedores como tecnología necesaria de base, pasamos a
estudiar distintas soluciones del mercado para implementar un sistemas de
orquestación basado en microservicios para el despliegue de aplicaciones de carácter
corporativo. Finalizamos con la implantación, instalación y puesta en marcha del
sistema estudiado acompañado de unos ejemplos de orquestación usando dos
aplicaciones de código abierto que se ven bastante habitualmente en los entornos
corporativos actuales para dar soporte a distintas soluciones.
\\
\textbf{Palabras clave:}   Virtualizacion y Contenedores .\\

\begin{center}
\textbf{Abstract}
\end{center}
In this job we do a study of container virtualization technologies will be carried out in
order to implement a system that allows orchestrating the deployment of applications
and services in enterprise environments.
This will be done in the first instance an analysis of the most common virtualization
systems to continue introducing concepts and systems of containerization
virtualization. Once the theoretical part is introduced, different container virtualization
tools are analyzed, focusing on the Docker tool, which details its architecture, operation
and installation process to finish with a couple of practical examples of service
deployment.
Then, once we have already implemented and analyzed a container virtualization
system and all necessary background technology, we started to study different
solutions in the market to implement a micro-services based orchestration systems for
the deployment of corporate applications.
Then finish with timplementation, installation and first steps of the studied system
accompanied by some examples of orchestration using two open source applications
that are quite commonly used in the current corporate environments to support
different solutions.
\\
\textbf{Keywords:}   Virtualization and Containers

\end{abstract}



\maketitle

%\tableofcontents

\section {Introducción}\label{sec:1}
La tecnología que ha llegado para complementar y completar la virtualización de servidores es la
virtualización de contenedores de aplicaciones. Esta tecnología va un paso mas allá en el
paradigma de la virtualización, permitiendo no sólo el salto de virtualizar servidores sino también
de virtualizar directamente un contenedor donde se ejecuta una aplicación, permitiendo de este
modo una mayor abstracción aislando la componente "lógica de la aplicación" del componente
“sistema operativo”.


%-----------------------------------------------------------------
\section{Objetivos}\label{sec:2}

\subsection{General:}

-  La puesta en marcha de una solución que permita la Orquestación de aplicaciones corporativas a través
de la implantación de un sistema de contenedores virtuales usando la plataforma abierta Docker 
\subsection{Específicos:}
- Reducción del gasto en infraestructura de la empresa para el despliegue de aplicaciones
corporativas

%-----------------------------------------------------------------
\section {Marco Teórico}\label{sec:3}


\subsection{Máquinas Virtuales}
Una de las tecnologías principales que han propiciado la consolidación del cloud es la
virtualización, consistente en la creación o simulación a través de software de una versión virtual
de algún recurso tecnológico. Este recurso suele ser habitualmente CPU, Memoria,
Almacenamiento y componentes de Red (o la combinación de varios) y la capa de software que
los gestiona y maneja, conocida como Hypervisor o VMM (Virtual Machine Monitor), que crea una
capa de abstracción entre estos recursos y el sistema operativo de las máquinas “invitado” (como
se conocen los sistemas operativos virtuales que se ejecutan sobre el mismo anfitrión y sobre el
que comparten recursos).
Dicho de otra manera y en un sentido más general, virtualizar es particionar un servidor físico en
varios servidores virtuales compartiendo todos ellos los recursos hardware de la máquina anfitrión.

\subsection{Contenedores}
La virtualización basada en contenedores es una aproximación a la virtualización en la cual se
ejecuta la capa de virtualización como una aplicación aislada dentro del sistema operativo del
equipo anfitrión. En este tipo de sistemas sólo se ejecuta un único núcleo del sistema operativo o
Kernel19, que es el del sistema operativo anfitrión, y este ayudado por el software de virtualización
de contenedores específico es el que se encarga de crear nuevos entornos de ejecución (que
podríamos comparar con las máquinas virtuales) y que reciben el nombre de contenedores.
Un contenedor es sencillamente un proceso para el sistema operativo que internamente contiene
la aplicación que queremos ejecutar y todas las dependencias derivadas de la misma.
Empaquetamos una aplicación en una unidad estandarizada para desempeñar un servicio que
contiene lo necesario para funcionar como un todo (código, librerías, software, etc), empaquetado
bajo la analogía de un contenedor.

\subsection{Diferencias}

%-----------------------------------------------------------------

%-----------------------------------------------------------------
\section{tipos de Virtualizacion}\label{sec:4}
\subsection{¿Cuales son las Ventajas de la Virtualizacion}
-  
\subsection{Contenedores}
- 

%-----------------------------------------------------------------
\section {Virtualizacion de Contenedores}\label{sec:5}

La virtualización basada en contenedores es una aproximación a la virtualización en la cual se
ejecuta la capa de virtualización como una aplicación aislada dentro del sistema operativo del
equipo anfitrión. En este tipo de sistemas sólo se ejecuta un único núcleo del sistema operativo o
Kernel19, que es el del sistema operativo anfitrión, y este ayudado por el software de virtualización
de contenedores específico es el que se encarga de crear nuevos entornos de ejecución (que
podríamos comparar con las máquinas virtuales) y que reciben el nombre de contenedores.
Un contenedor es sencillamente un proceso para el sistema operativo que internamente contiene
la aplicación que queremos ejecutar y todas las dependencias derivadas de la misma.
Empaquetamos una aplicación en una unidad estandarizada para desempeñar un servicio que
contiene lo necesario para funcionar como un todo (código, librerías, software, etc), empaquetado
bajo la analogía de un contenedor.

\subsection{Diferencia Entre Virtualizacion Clasica y Virualizacion de Contenedores}
\subsection{Tecnologia a Utilizar}
\subsection{Presedentes a Utilizar}
\subsection{Ventajas y Desventajas}

%-----------------------------------------------------------------

%-----------------------------------------------------------------
\section{Contenedores Docker}\label{sec:6}


Además de la propia herramienta principal de Docker para el control y gestión de los
contenedores, existe un importante ecosistema de aplicaciones y utilidades integrado en el propia
Docker que permite dotar a esta solución de un gran numero de herramientas y funcionalidades
adicionales que aportan distintas soluciones de valor. Docker Machine, Docker Swarm y Docker
Compose son tres de las más salientables y que combinadas, permiten que los contenedores
sean más portables y escalables de tal forma que pueden ser más fácilmente desplegados y
administrados en conjunto por lo que combinadas permiten la orquestación de sistemas a gran
escala. Así, todo el ecosistema de aplicaciones que aporta la solución Docker van orientadas a
dotar de un mayor numero de funcionalidades y herramientas de gestión y automatización para un
sistema de aplicaciones embebidas en un contenedor virtual.
Podemos considerar que con Docker podemos realizar empaquetados de aplicaciones como si se
tratase de una unidad estandarizada y estanca (contenedor) que incluya todo lo necesario para
ejecutar esa aplicación (Código, librerías, bibliotecas del sistema, herramientas, etc) y que puede
ser replicado de forma rápida, fiable y sistemática.

\subsection{Orquesta de Aplicacion}

\subsection{Docker y Otros Container : Mas alla de la virtualizacion}
- Docker es una aplicación que sigue el modelo cliente-servidor. Así, a nivel de arquitectura de la
aplicación tenemos fundamentalmente una parte de servidor denominada “Docker Server” y una
aplicación cliente, denominada “Docker Client” que hace uso de funcionalidades de la parte
servidor, fundamentalmente a través de un cliente en línea de comandos. El modo servidor de
Docker es el que se encarga de toda la gestión, mantenimiento y creación de los contenedores y
su estructura de funcionamiento. La aplicación es única para ambos modos, sólo que al ejecutarla
se indica si lo que queremos arrancar es el modo cliente o el modo servidor.
Existe también un tercer componente denominado, Docker Registry. Se trata de un repositorio
donde se albergan las imágenes de los contenedores virtuales a emplear.
Mediante el cliente Docker se envían ordenes al servidor Docker que es el que gestiona el
funcionamiento del sistema base de contenedores y este a su vez, mediante el uso de la librería
libcontainer interactúa directamente con las distintas utilidades del Kernel para acceder a las
utilidades de virtualización. Aunque libcontainer es actualmente el método nativo de interacción
con el Kernel, Docker puede usar otras librerías o interfaces intermedias, como pueden ser libvirt o
LXC.
\subsection{Contenedor Docker la Tecnologia de Contenedores a mano}
\subsection{Ventajas de la Tecnologia de Contenedores}
\subsection{¿Son Seguros los Contenedores?}

%-----------------------------------------------------------------
\section {Hypervisores Base Metal}\label{sec:7}


\subsection{Tipos de Hypervisores}
\subsection{Ventajas y Desventajas}
\subsection{Productos Comerciales}

%-----------------------------------------------------------------
\section{Conclusiones}\label{sec:8}


\begin{itemize}
\item Como hemos podido comprobar a lo largo de es proyecto, partimos de los requisitos base de una
empresa que tenia un centro de datos con servidores propios, con personal para mantenerlo y que
pretendía encontrar y validar un sistema de código abierto que les permitiese implantar un sistema
de orquestación de aplicaciones.
Como ha quedado patente, se han conseguido todos los objetivos iniciales que se planteaban al
inicio del mismo.
Para ello, en primer lugar se ha realizado un análisis de las distintas tecnologías de virtualización
clásica y de virtualización de contenedores, comparándolas y analizando cual era la solución por
la que debíamos optar para la orquestación de aplicaciones. Una vez visto que la solución pasaba
por el uso de la virtualización de contenedores, se estudio la tecnología subyacente que la hacia
posible, para poder entender su funcionamiento y posibles limitaciones. A continuación se
analizaron varias alternativas de aplicaciones de virtualización y orquestación, con la premisa
inicial del uso de software de código abierto, que nos permitieran acometer una solución, hasta
decantarnos finalmente por la herramienta Docker y su ecosistema de aplicaciones como solución
para conseguir los objetivos planteados.
Al mismo tiempo que se selecciono la aplicación, se planteo una arquitectura complementaria para
el despliegue de la solución Docker.
Y llegados a este punto, se realizo un análisis exhaustivo del funcionamiento de todo el
ecosistema de aplicaciones de Docker, analizando y estudiando su funcionamiento y
documentándolo con ejemplos prácticos.
Considerando que existen dos tipos de orquestaciones, la orquestación estática y la orquestación
dinámica, analizamos en profundidad dos de las herramientas del ecosistema Docker que
permiten dar solución a los objetivos que nos planteábamos: Comprobamos como Docker
Compose nos permitía orquestar servicios de manera estática y como Docker Swarm nos permitía
lo mismo pero esta vez para orquestados dinámicamente con lo cual, la solución propuesta
cumple todas la condiciones iniciales que nos marcamos.
Finalmente se realizaron dos ejemplos prácticos completos de cada una de las distintas
orquestaciones, cada una con su herramienta particular (Docker Compose y Docker Swarm)
pudiendo verificar que la elección de la solución Docker nos permitió dar una solución completa
que incluso dotaba a nuestro proyecto de posibilidades de escalado y orquestación con una
sencillez más allá de las esperadas personalmente al comienzo del mismo.

\end{itemize}

\section{webgrafia}\label{sec:9}
% Bibliografia.
%-----------------------------------------------------------------

\bibliographystyle{plain}
\bibliography{Bibliografia}

\end{document}