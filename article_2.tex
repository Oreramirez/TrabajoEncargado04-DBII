%%%%%%%%%%%%%%%%%%%%%%%%%%%%%%%%%%%%%%%%%
% Journal Article
% LaTeX Template
% Version 1.4 (15/5/16)
%
% This template has been downloaded from:
% http://www.LaTeXTemplates.com
%
% Original author:
% Frits Wenneker (http://www.howtotex.com) with extensive modifications by
% Vel (vel@LaTeXTemplates.com)
%
% License:
% CC BY-NC-SA 3.0 (http://creativecommons.org/licenses/by-nc-sa/3.0/)
%
%%%%%%%%%%%%%%%%%%%%%%%%%%%%%%%%%%%%%%%%%

%----------------------------------------------------------------------------------------
%	PACKAGES AND OTHER DOCUMENT CONFIGURATIONS
%----------------------------------------------------------------------------------------

\documentclass[twoside,twocolumn]{article}

\usepackage{blindtext} % Package to generate dummy text throughout this template 

\usepackage[sc]{mathpazo} % Use the Palatino font
\usepackage[T1]{fontenc} % Use 8-bit encoding that has 256 glyphs
\linespread{1.05} % Line spacing - Palatino needs more space between lines
\usepackage{microtype} % Slightly tweak font spacing for aesthetics

\usepackage[english]{babel} % Language hyphenation and typographical rules

\usepackage[hmarginratio=1:1,top=32mm,columnsep=20pt]{geometry} % Document margins
\usepackage[hang, small,labelfont=bf,up,textfont=it,up]{caption} % Custom captions under/above floats in tables or figures
\usepackage{booktabs} % Horizontal rules in tables

\usepackage{lettrine} % The lettrine is the first enlarged letter at the beginning of the text

\usepackage{enumitem} % Customized lists
\setlist[itemize]{noitemsep} % Make itemize lists more compact

\usepackage{abstract} % Allows abstract customization
\renewcommand{\abstractnamefont}{\normalfont\bfseries} % Set the "Resumen" text to bold
\renewcommand{\abstracttextfont}{\normalfont\small\itshape} % Set the abstract itself to small italic text

\usepackage{titlesec} % Allows customization of titles
\renewcommand\thesection{\Roman{section}} % Roman numerals for the sections
\renewcommand\thesubsection{\roman{subsection}} % roman numerals for subsections
\titleformat{\section}[block]{\large\scshape\centering}{\thesection.}{1em}{} % Change the look of the section titles
\titleformat{\subsection}[block]{\large}{\thesubsection.}{0.5em}{} % Change the look of the section titles

\usepackage{fancyhdr} % Headers and footers
\pagestyle{fancy} % All pages have headers and footers
\fancyhead{} % Blank out the default header
\fancyfoot{} % Blank out the default footer
\fancyhead[C]{VIRTUALIZACIÓN Y CONTENEDORES $\bullet$ Mayo 2019 $\bullet$ Trabajo, Nro. 4} % Custom header text
\fancyfoot[RO,LE]{\thepage} % Custom footer text

\usepackage{titling} % Customizing the title section

\usepackage{hyperref} % For hyperlinks in the PDF

%----------------------------------------------------------------------------------------
%	TITLE SECTION
%----------------------------------------------------------------------------------------

\setlength{\droptitle}{-4\baselineskip} % Move the title up

\pretitle{\begin{center}\Huge\bfseries} % Article title formatting
\posttitle{\end{center}} % Article title closing formatting
\title{VIRTUALIZACIÓN Y CONTENEDORES} % Article title
\author{%
\textsc{Marko Antonio Rivas Rios} \\[1ex] % Your name
\textsc{Jorge Luis Mamani Maquera} \\[1.01ex] % Your name
\textsc{Orlando Antonio Acosta Ortiz} \\[1.02ex] % Your name
\textsc{Yofer Nain Catari Cabrera} \\[1.03ex] % Your name
\textsc{Orestes Ramirez Ticona} \\[1.04ex] % Your name
\textsc{Roberto Zegarra Reyes} \\[1.05ex] % Your name
\normalsize Universidad Privada de Tacna \\  % Your institution
\normalsize {} % Your email address
%\and % Uncomment if 2 authors are required, duplicate these 4 lines if more
%\textsc{Jane Smith}\thanks{Corresponding author} \\[1ex] % Second author's name
%\normalsize University of Utah \\ % Second author's institution
%\normalsize \href{mailto:jane@smith.com}{jane@smith.com} % Second author's email address
}
\date{Mayo 13, 2019} % Leave empty to omit a date
\renewcommand{\maketitlehookd}{%
\begin{abstract}
\noindent En este trabajo se realizará un estudio de las tecnologías de virtualización de
contenedores con el fin de implementar y poner en marcha un sistema que permita
orquestar el despliegue de aplicaciones sobre un entorno empresarial.
Para ello, se realizará en primera instancia un análisis de los sistemas de virtualización
más habituales para continuar introduciendo los conceptos y sistemas de virtualización
de contenedores. Una vez introducida la parte teórica se analizan distintas
herramientas de virtualización de contenedores centrándonos en la herramienta
Docker para la cual se detalla su arquitectura, funcionamiento y proceso de instalación
para finalizar con un par de ejemplos prácticos de despliegue de servicios.
A continuación, una vez que ya hemos implementado y analizado un sistema de
virtualización de contenedores como tecnología necesaria de base, pasamos a
estudiar distintas soluciones del mercado para implementar un sistemas de
orquestación basado en microservicios para el despliegue de aplicaciones de carácter
corporativo. Finalizamos con la implantación, instalación y puesta en marcha del
sistema estudiado acompañado de unos ejemplos de orquestación usando dos
aplicaciones de código abierto que se ven bastante habitualmente en los entornos
corporativos actuales para dar soporte a distintas soluciones.
\end{abstract}
}

%----------------------------------------------------------------------------------------

\begin{document}

% Print the title
\maketitle

%----------------------------------------------------------------------------------------
%	ARTICLE CONTENTS
%----------------------------------------------------------------------------------------

\section{Introducción}

\lettrine[nindent=0em,lines=2]{L}a tecnología llego para complementar y completar la virtualización de servidores es la
virtualización de contenedores de aplicaciones. Esta tecnología va un paso mas allá en el
paradigma de la virtualización, permitiendo no sólo el salto de virtualizar servidores sino también
de virtualizar directamente un contenedor donde se ejecuta una aplicación, permitiendo de este
modo una mayor abstracción aislando la componente "lógica de la aplicación" del componente
“sistema operativo”.


\section{Objetivos}
\begin{flushright}
\begin{itemize}
\lettrine[nindent=0em,lines=2]{E}l desarrollo

%------------------------------------------------


\section{Virtualización y Contenedores}

\subsection{Virtualización}


 \textbf{A). Tipos de Virtualizacion }\\
\textbf{}\\
\textbf{}\\
\textbf{}\\
 \textbf{B). Cuales son las Ventajas de la Virtualización}\\





\subsection{Contenedores}
 \textbf{A). Virtualización de Contenedores }\\
\textbf{}\\
\textbf{}\\
\textbf{}\\
 \textbf{B). Diferencia entre Virtualizacion Clasica y Virtualizacion de Contenedores}\\
\textbf{}\\
 \textbf{C).Tecnologias a Utilizar}\\
\textbf{}\\
\textbf{}\\
 \textbf{D).Ventajas y Desventajas}\\
\textbf{}\\



\subsection{Cotenedores con Docker}
Docker es una plataforma de software que le permite crear, probar e implementar aplicaciones rápidamente. Docker empaqueta software en unidades estandarizadas llamadas contenedores que incluyen todo lo necesario para que el software se ejecute, incluidas bibliotecas, herramientas de sistema, código y tiempo de ejecución. Con Docker, puede implementar y ajustar la escala de aplicaciones rápidamente en cualquier entorno con la certeza de saber que su código se ejecutará.

La ejecución de Docker en AWS les ofrece a los desarrolladores y administradores una manera muy confiable y económica de crear, enviar y ejecutar aplicaciones distribuidas en cualquier escala. AWS es compatible con ambos modelos de licencia de Docker: Docker Community Edition (CE) de código abierto y Docker Enterprise Edition (EE) basada en suscripción.

Docker, me permite meter en un contenedor (“una caja”, algo auto contenido, cerrado) todas aquellas cosas que mi aplicación necesita para ser ejecutada (java, Maven, tomcat…) y la propia aplicación. Así yo me puedo llevar ese contenedor a cualquier máquina que tenga instalado Docker y ejecutar la aplicación sin tener que hacer nada más, ni preocuparme de qué versiones de software tiene instalada esa máquina, de si tiene los elementos necesarios para que funcione mi aplicación , de si son compatibles.

 \textbf{A). Orquesta de Aplicación }\\
\textbf{}Existen múltiples definiciones sobre el concepto de orquestación de aplicaciones pero de un modo simple, podemos definir la orquestación de servicios o aplicaciones como el uso de la automatización para la creación y composición de la arquitectura, herramientas y procesos utilizados por operadores humanos para entregar un servicio.\\
\textbf{}La orquestación aprovecha tareas automatizadas y procesos predefinidos para permitir la creación de infraestructura complejas y para conseguir el aprovechamiento de los recursos de forma óptima y automatizada. Podemos considerar, a modo de analogía, el concepto de orquestación como un proceso y la automatización como una tarea.\\
\textbf{}De este modo, el objetivo principal de la orquestación consiste en la automatización de procesos orientados al despliegue y ciclo de vida de las aplicaciones o servicios. Y la automatización de procesos en los despliegues software se basan en el uso de algún tipo de software que facilite la\\
 \textbf{B). Docker y otros container: más alla de la virtualizacion}\\
\textbf{}En un mundo donde cada vez es más común el uso de servicios informáticos en la nube y cuyos principios básicos para explicar su éxito radican en la alta disponibilidad, el diseño de tolerancia a fallos, el escalado y la elasticidad, parece que se hace obligatorio que hablemos sobre los contenedores. En este post nos referimos en concreto al proyecto “Docker”.\\
\textbf{}Cuando pensábamos que ya habíamos alcanzado el máximo ahorro de recursos con la virtualización de todo el hardware posible de nuestra infraestructura (servidores, redes, cabinas de discos, etc.), desde hace poco tiempo se ha extendido el uso de los llamados contenedores. La idea de esta post es repasar los conceptos básicos que rodean al mundo de los contenedores y dar una visión general sobre esta herramienta.\\
 \textbf{C). Contenedor Docker, la Tecnología de Contenedores a Mano}\\
\textbf{}Para hablar de futuro, es necesario observar el camino recorrido y los avances tecnológicos no son ajenos a ellos. Si hablamos de virtualización 3.0, se podría decir que inició desde el uso de los primeros mainframes los cuales permitían que varios usuarios operaran al mismo tiempo a través de los “terminales tontos”.\\
\textbf{}\\
 \textbf{D).¿Son Seguros los Contenedores}\\
\textbf{}Podemos pensar en un contenedor como un servidor que arranca desde una imagen estática predefinida con un sistema operativo con un kernel Linux y con las librerías y recursos mínimos necesarios de CPU, memoria, almacenamiento, etc. Realmente, un contenedor consiste en el empaquetado de una aplicación para que pueda correr en cualquier sistema abstrayéndose de la plataforma sobre la que está corriendo.\\



\subsection{Hypervisores Bare Metal}

 \textbf{A). Tipos de Hypervisores }\\
\textbf{}\\
\textbf{}\\
\textbf{}\\
 \textbf{B). Ventajas y Desventajas}\\
\textbf{}\\
 \textbf{C). Productos Comerciales}\\
\textbf{}\\
\textbf{}\\



\section{Conclusiones}





\textbf{}\\
\textbf{}\\
%----------------------------------------------------------------------------------------
%	REFERENCE LIST
%----------------------------------------------------------------------------------------

\begin{thebibliography}{99} % Bibliography - this is intentionally simple in this template

\bibitem[]{}

\newblock 
http://revistatelematica.cujae.edu.cu/
index.php/tele/article/view/23/21
\break
https://programarfacil.com/blog/
que-es-un-orm/




\newblock {\em }
 
\end{thebibliography}

%----------------------------------------------------------------------------------------
\end{itemize}
\end{flushright}
\end{document}

